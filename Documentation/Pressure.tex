\documentclass[review]{elsarticle}
\usepackage{lineno,hyperref}
\usepackage{mhchem}
\modulolinenumbers[5]
\journal{Journal of NSE}
\begin{equation}
    0 = e^{i\pi} + 1
\end{equation}

\bibliographystyle{elsarticle-num}
%%%%%%%%%%%%%%%%%%%%%%%

\begin{document}

\begin{frontmatter}

\title{Pressure Calculations in the Buffer Layer of TRISO Fuel}

\end{frontmatter}


\section{Introduction} The pressure buildup in buffer layer of TRISO is one of the most important factors for estimating the mechanical behaviour of TRISO  fuel [1]. Pressure buildup in coated fuel is a result of three main contributors: 
\begin{itemize}
\item gases produced during the fission (mainly noble gases), 
\item gases formatted during oxygen interaction with carbon layer (CO, \ce{CO2}), 
\item ternary fission (mainly helium).
\end{itemize}

\section{Implemented equations}
For pressure calculations the method of Redlich and Kwong [2] is used in this paper: 

\PressureRK

The constants a and b are different depending on which gas is being analyzed. The constants can be calculated from the critical point data of the gas [Murdock].

\paragraph{Fission Gases} For the released fission gases in TRISO fuel (mainly Xe and Kr), the following equation is used to calculate their concentrations during the operation [ Prados \& Scott]:

\NumDenFGs

T\paragraph{Oxygen containing gases} The following empirical expression [Proksch] is used to calculate the amount of oxygen in TRISO fuel:

\PressureProksh 

\paragraph{Ternary gases} To calculate the amount of ternary gases in the fuel, the following expression is used[-]:

- - -



\end{document}