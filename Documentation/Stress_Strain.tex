\documentclass[review]{elsarticle}
\usepackage{lineno,hyperref}
\usepackage{mhchem}
\modulolinenumbers[5]
\journal{Journal of NSE}
\begin{equation}
    0 = e^{i\pi} + 1
\end{equation}

\bibliographystyle{elsarticle-num}
%%%%%%%%%%%%%%%%%%%%%%%

\begin{document}

\begin{frontmatter}

\title{Stress and Strain Calculations in TRISO Fuel}

\end{frontmatter}


\section{Introduction} To analyze  the behavior of TRISO fuel one should perform mechanical and thermal calculations including all occurring phenomena during the fuel operation. The main indicator of fuel safety criteria is the max strength of SiC layer of the particle. In other words, TRISO is called failed, when the stress acting in SiC layer is higher than the critical stress value. To calculate the overall stress on SiC layer, the following contributors should be addressed: 
\begin{itemize}    
    \item Elastic deformation,
    \item Thermal expansion,
    \item Irradiation induced creep,
    \item Irradiation induced dimensional changes (swelling).
\end{itemize}
Each of these components are dependent in many factors, such as pressure buildup, fuel kernel migration, anisotropy of graphite, temperature distribution in fuel and etc.

\section{Implemented equations}
The general strain in the system is equal to:
\StrainGeneral

The relationship between strain and displacement in radial and tangential directions for spherical coordinates is given by:
\StrainAndDisplacement

\paragraph{Elastic Strain} The relationship between radial and tangential elastic strains and stresses in spherical coordinate system  is as follows:
\StrainElastic

\paragraph{Creep Strain} The relationship between radial and tangential creep strains and stresses in spherical coordinate system  is as follows: 
\StrainCreep

\paragraph{Swelling Strain} The radial and tangential swelling strains are given by  empirical expression: 
For different cases they are different. Here is presented one of the correlations used in benchmark:

Correlation a)
\StrainSwelling


\paragraph{Thermal Strain} The radial and tangential thermal strains are calculated using the following formula:
\StrainThermal 

\paragraph{General equation} Using all the equations mensioned obove, we can obtain the general form of stress-displacement relationship for radial and tangential directions:

\GeneralStrDisRadial
\GeneralStrDisTangential

\paragraph{Equilibrum equation} The relations between radial and tangential stresses in equilibrium state for the spherical geometries is given by:
\EquilibrumStress 

\paragraph{Final equation} 
Now, using general and equilibrum equations, we can get formulas for radial and tangential stresses. For this purpose, first, we should differenciate the general equation with respect to flux, in order to get rid of integrals:
\EquationsForFinRad
\EquationsForFinTan

To solve this set of equations with equilibrum equation, we will make the following general series assumptions:
\DisplaceSeries
\StressSeries
\StrainSeries


Putting all in main set of equations we will get:
\MainSetRadOne
\MainSetRadTwo

\MainSetTanOne
\MainSetTanTwo

For zero term $\phi^0$ we have then:
\MainSetTermZeroRad
\MainSetTermZeroTan

Now, multypling (16)$\cdot$(1-$\mu$)  + (17)$\cdot$2 $\mu$ we can have:
\EqEighteenPone
\EqEighteenPTwo

Differenciating (18) with respect to r, we can get:
\EqNineteenPOne
\EqNineteenPTwo

Using equilibrum (8)  and (19) equations we will recive:
\EqTwentyPOne
\EqTwentyPTwo

From (16) - (17) we also have:
\EqTwentyOne	

Substituting $(\sigma_{r1} - \sigma_{\theta1})$ from (21) and putting in (20) we can recive: 
\EqTwentyTwo

It can be proved, that$\Big(\frac{1}{2} \frac{d\sigma_{r0}}{dr} + \frac{d\sigma_{\theta0}}{dr}\Big) =0 $,
so from (22) we can get:
\EqTwentyThree

Equation (23) is Euler's equation and its standard solution is:
\EqTwentyFour


Now, putting equation (23) into equations  (18) and (21) we inally can get:
\EqTwentyFivePOne
\EqTwentyFivePTwo 	

and
\EqTwentySix

Now, when we get the solution for $\phi^{0}$ term, we can follow the same procedure for $\phi^{i}$ term.
Based on (16) and (17) equations, we have:
\MainSetTermIRad
\MainSetTermITan

And then like (22):
\EqTwentyNine

When i=1, based on equations (25) and (26), we can have:
\EqThirty

Lets make the following notations:
\EqThirtyOne
\EqThirtyTwo

Now from (29) we can obtain:
\EqThirtyThree

So as we did in case of equation (23), the solution of (32) will look like this:
\EqThirtyFour

Following the same procedure as in previous case, we can obtain:
\EqThirtyFivePOne
\EqThirtyFivePTwo 	

and
\EqThirtySix

So we have already solutions for both $\phi^{0}$ and $\phi^{1}$ terms. Following the same analogy, we can find the recursive relationship for term $\phi^{i}$. Similar to previuos cases we will heve:
\EqThirtySeven
\EqThirtyEigth
\EqThirtyNine
and finally:
\EqFourty

Now, from equation (40) we can find $u_{i+1}(r) $, then using equations (26) and (27) we will obtain   
$\sigma_{r i+1} (r)$ and $\sigma_{\theta i+1}(r)$. Total strain values can be found usinq equation (2).
To sum up, for any term of $\phi^{i}$ we have:
\EqFinDisplSol
\EqFinalStrRadSolOne
\EqFinalStrRadSolTwo
\EqFinalStrTanSolOne
\EqFinalStrTanSolTwo
\EqFinalStrainRadSol
\EqFinalStrainTanSol

As was mentioned before, $A_{i}$ and $B_{i}$ should be determined from boundary conditions, $f_{i}$ can be obtained using recursive relation (39). To solve the set of these equations, we will need aslo initial conditions for our system, which means the initial values of stress, strain and displacements.


\end{document}